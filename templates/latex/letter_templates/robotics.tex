I did my undergrad in mechatronics engineering in Canada, and my masters in a robotics lab in Kyoto. I have worked in a variety of fields in four countries. I enjoy automating tasks (including using hardware), as well as getting useful insights out of data.

My primary language for the last seven years have been Python, where I focused primarily on the scipy stack. (I even have two packages on PyPI.) I have proven experience in delivering software projects which are well-documented, robust, modular, and maintainable. I have been using Linux as my primary OS since 2009.

I am also no stranger to hardware. I have hands-on experience with electronics, including microcontrollers, soldering, and test benches. I have worked with robots and sensors, and have written my own device drivers when they did not exist. In my most recent job as an Industry 4.0 consultant, I talked with a laser profiler and camera in order to monitor a conveyor belt using Python and ROS.

During my undergrad, my main focus was on the physical aspect of robotics. My internships included research, electronics, and the design of industrial machines. During my masters, I programmed and controlled robots and drones, using both my own infrastructure and with ROS. I also developed the interface for the drones, ran analyses, and produced visualizations.
